\section{Приложение математического\\моделирования для исследования\\экономических процессов}

В экономической части настоящей работы будет рассмотрена ранняя \textbf{модель капитализации Калецкого (Kalecki)}. Данная модель \--- одно из первых распространенных приложений дифференциальных уравнений с запаздыванием. Впервые описание данной модели самим Калецким было представлено в работе \cite{bib:Kalecki}.

\subsection{Основное макроэкономическое тождество}

Исходым допущением в модели Калецкого является выполнение \textbf{основного макроэкономического тождества}:

\begin{equation}
Y = C + I + A,
\end{equation}

где

\begin{enumerate}
\item $Y$ \--- совокупный доход (часто в его качестве рассматривается \textbf{валовый национальный продукт}(ВНП)),
\item $C = C_a + C_i$ \--- сумма интенсивностей автономного и индуцированного потребления,
\item $C_i = c Y$ \--- индуцированное потребление (т.е часть расходов, обусловленная изменением уровня располагаемого дохода $Y$),
\item $C_a$ \--- автономное потребление (т.е. часть расходов, постоянная вне зависимости от величины доходов $Y$),
\item $I = I_a + I_i$ \--- совокупные инвестиции,
\item $I_i = s Y$ \--- индуцированные инвестиции,
\item $I_a$ \--- автономные инвестиции,
\item $c+s=1, c>0, s>0$.
\end{enumerate}

Оно отражает равенство дохода и совокупных расходов (а значит, истинно только для \texbf{замкнутной экономической системы}).

\subsection{Вывод дифференциального уравнения с запаздыванием}

Пусть $K(t)$ \--- объем доступного капитала. Его динамику можно рассчитать следующим образом в силу запаздывания реакции системы на новый капитал:

\begin{equation}
K'(t + \tau) = a s Y(t) - b K(t) + \epsilon(t), \quad 0 < a < 1, b > 0,
\end{equation}

$\epsilon(t)$ \--- невязка.

Инвестиции $I(t)$ \--- изменение доступного капитала в атомарный период времени:

\begin{equation}
I(t) \dfrac{1}{\tau} \left( K(t+\tau) - K(t) \right)
\end{equation}

Далее

\begin{equation}
Y(t) = underbrace{(1-s)) Y(t)}_{C_a(t)}}_{C(t)} + I(t) + A(t)
\end{equation}

\begin{equation}
Y(t) = \dfrac{1}{s} \left( I(t) + A(t) + C_a(t) \right)
\end{equation}

\begin{equation}
Y(t) = \dfrac{1}{s \tau} \left( K(t+\tau) - K(t) \right) + \dfrac{1}{s} \left( A(t) + C_a(t) \right)
\end{equation}

Таким образом,

\begin{equation}
K'(t + \tau) = a \dfrac{1}{\tau} \left( K(t+\tau) - K(t) \right) + a \left( A(t) + C_a(t) \right) - bK(t) + \epsilon(t)
\end{equation}

Наконец, перепишем уравнение в виде:

\begin{equation}
K'(t) - \dfrac{a}{\tau} K(t) + \left( \dfrac{a}{\tau} + b \right) K(t-\tau) = a (A(t-\tau) + C_a(t-\tau) + \epsilon(t-\tau))
\end{equation}