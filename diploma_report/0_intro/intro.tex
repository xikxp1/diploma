\section*{Введение}
\addcontentsline{toc}{section}{Введение}

Данный дипломный проект посвящен изучению математических моделей теплопроводности с отклоняющимся аргументом. 

Впервые теорию дифференциальных уравнений сначала с запаздывающим (а потом и с отклоняющимся) аргументом рассмотрел в А.Д.Мышкис в работе \cite{bib:Myshkis-main}). Другой пионерской работой в исследовании этого вопроса стала \cite{bib:Hahn}.

В последствии дифференциальные уравнения с отклоняющимся аргументом (также иногда называемые функционально-дифференциальными) нашли многочисленные приложения в различных вопросах автоматики, в теории колебаний, в ракетной технике, во многих вопросах физики, в некоторых задачах экономических, биологических и медицинских наук. Важность и разнообразие приложений резко повысили интерес к теории этих уравнений. В 60-х годах XX века появилось множество работ по данной тематике: в качестве примеров можно привести монографии Белммана, Данскина и Гликсберга \cite{bib:Bellman-co}, Беллмана и Данскина \cite{bib:Bellman-Danskin}, Пинни \cite{bib:Pinny}, Красовского \cite{bib:Krasovskii} и др.

Теория дифференциальных уравнений с отклоняющимся аргументом до сих пор находится в стадии становления и интерес к ней по-прежнему не угас. Значимыми работами современности по данной тематике стали \cite{bib:Wu} и \cite{bib:Kolmanovskii-Myshkis}.

В данной работе будет исследована устойчивость уравнения локального закона сохранения (Лиувилля) с малым запаздыванием $\tau$:

\begin{equation*}
\left\{
\begin{aligned}
\rho_t(x,t) + div \bar{J}(x,t) = f(x,t),\\
\bar{J}(x,t+\tau) = -a^2 \nabla \rho(x,t) + \bar{V}(x,t) \rho(x,t)
\end{aligned}
\right.
\end{equation*}

Цель настоящего исследования \--- оценить, как небольшое отклонение по времени влияет на решения и их устойчивость для уравнений теплопроводности с отклоняющимся аргументом, а также сравнить моделей с отклонением и без. Также в данной работе будет приведена необходимая теория и представлены развитые численные методы решений дифференциальных уравнений с отклоняющимся аргументом.

Важным фактом является наличие именно малого запаздывания: в работе будет представлен переход от исходной модели к системе обыкновнных дифференциальных уравнений с малым параметром при старшей производной. Такие уравнения были рассмотрены в статьях Тихонова \cite{bib:Tikhonov_1, bib:Tikhonov_2}, Васильевой \cite{bib:Vasilieva}, в книге Васильевой и Бутузова \cite{bib:Vasilieva-Butuzov}.

В экономической части работы будет \textbf{модель капитализации Калецки (Kalecki)}\cite{bib:Kalecki}.