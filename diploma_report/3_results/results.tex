\section{Результаты проведённого анализа}

Цель данного раздела настоящей работы \--- сравнение решений обыкновенного дифференциального уравнения с запаздыванием

\begin{equation}\label{eq:result-ODE-delay}
{T'}_{k}(t) - \gamma(k^2) T_k(t-\tau)=0
\end{equation}

и его приближений обыкновенными дифференциальными уравнениями уже без запаздывания:

\begin{equation}\label{eq:result-ODE-no-delay}
\sum\limits_{n=1}^{m} \dfrac{\tau^{n-1}}{(n-1)!} T_k^{(n)} (t) + \gamma(k^2) T_k (t) = 0
\end{equation}

В предыдущих главах было доказано, что достаточным условием устойчивости (\ref{eq:result-ODE-delay}) является соотношение $\gamma \tau <1$. Приближения (\ref{eq:result-ODE-no-delay}) же, вообще говоря, всегда неустойчивы при порядках $m \geq 6$ и неустойчивы в предположениях фиксированного малого $\tau$ для порядков $m \geq 2$. Приближение порядка $m=1$ всегда устойчиво.

В рамках работы была написана программа для построения графиков решений задач Коши для функций (\ref{eq:result-ODE-delay}) и (\ref{eq:result-ODE-no-delay}) в зависимости от параметров 

\begin{enumerate}
\item $\tau$ \--- величины запаздывания,
\item $k$ \--- порядка члена ряда Фурье,
\item $m$ \--- порядка приближения.
\end{enumerate}

Начальной функцией предполагается единичная функция на $[-\tau,0]$. Скорости $a$ и $v$ для простоты примем также единичными.

Будем обозначать как $T$ \- решение (\ref{eq:result-ODE-delay}) и как $Tm$ \--- решение \ref{eq:result-ODE-no-delay}.

\newpage

\subsection{Графики решений при $\tau=0.1$, $k=1$}\label{sec:graph_1}

\vfill

\begin{figure}[h]
\begin{center}
\includegraphics[width=0.9\textwidth]{./3_results/1_1.eps}
\end{center}
\caption{График для $T(t)$ при $\tau=0.1$, $k=1$, $m=1$}
\end{figure}

\vfill

\newpage

\subsubsection{Приближение порядка $m=1$}

\begin{figure}[h]
\begin{center}
\includegraphics[width=0.65\textwidth]{./3_results/1_2.eps}
\end{center}
\caption{График для $Tm(t)$ при $\tau=0.1$, $k=1$, $m=1$}
\end{figure}

\begin{figure}[h]
\begin{center}
\includegraphics[width=0.65\textwidth]{./3_results/1_3.eps}
\end{center}
\caption{Общий график при $\tau=0.1$, $k=1$, $m=1$}
\end{figure}

\newpage

\subsubsection{Приближение порядка $m=2$}

\begin{figure}[h]
\begin{center}
\includegraphics[width=0.65\textwidth]{./3_results/1_4.eps}
\end{center}
\caption{График для $Tm(t)$ при $\tau=0.1$, $k=1$, $m=2$}
\end{figure}

\begin{figure}[h]
\begin{center}
\includegraphics[width=0.65\textwidth]{./3_results/1_5.eps}
\end{center}
\caption{Общий график при $\tau=0.1$, $k=1$, $m=2$}
\end{figure}

\newpage

\subsubsection{Приближение порядка $m=3$}

\begin{figure}[h]
\begin{center}
\includegraphics[width=0.65\textwidth]{./3_results/1_6.eps}
\end{center}
\caption{График для $Tm(t)$ при $\tau=0.1$, $k=1$, $m=3$}
\end{figure}

\begin{figure}[h]
\begin{center}
\includegraphics[width=0.65\textwidth]{./3_results/1_7.eps}
\end{center}
\caption{Общий график при $\tau=0.1$, $k=1$, $m=3$}
\end{figure}

\newpage

\subsubsection{Приближение порядка $m=4$}

\begin{figure}[h]
\begin{center}
\includegraphics[width=0.65\textwidth]{./3_results/1_8.eps}
\end{center}
\caption{График для $Tm(t)$ при $\tau=0.1$, $k=1$, $m=4$}
\end{figure}

\begin{figure}[h]
\begin{center}
\includegraphics[width=0.65\textwidth]{./3_results/1_9.eps}
\end{center}
\caption{Общий график при $\tau=0.1$, $k=1$, $m=4$}
\end{figure}

\newpage

\subsubsection{Приближение порядка $m=5$}

\begin{figure}[h]
\begin{center}
\includegraphics[width=0.65\textwidth]{./3_results/1_10.eps}
\end{center}
\caption{График для $Tm(t)$ при $\tau=0.1$, $k=1$, $m=5$}
\end{figure}

\begin{figure}[h]
\begin{center}
\includegraphics[width=0.65\textwidth]{./3_results/1_11.eps}
\end{center}
\caption{Общий график при $\tau=0.1$, $k=1$, $m=5$}
\end{figure}

\newpage

\subsubsection{Приближение порядка $m=6$}

\begin{figure}[h]
\begin{center}
\includegraphics[width=0.65\textwidth]{./3_results/1_12.eps}
\end{center}
\caption{График для $Tm(t)$ при $\tau=0.1$, $k=1$, $m=6$}
\end{figure}

\begin{figure}[h]
\begin{center}
\includegraphics[width=0.65\textwidth]{./3_results/1_13.eps}
\end{center}
\caption{Общий график при $\tau=0.1$, $k=1$, $m=6$}
\end{figure}

\newpage

\subsection{Графики решений при $\tau=0.1$, $k=2$}\label{sec:graph_2}

\vfill

\begin{figure}[h]
\begin{center}
\includegraphics[width=0.9\textwidth]{./3_results/2_1.eps}
\end{center}
\caption{График для $T(t)$ при $\tau=0.1$, $k=2$, $m=1$}
\end{figure}

\vfill

\newpage

\subsubsection{Приближение порядка $m=1$}

\begin{figure}[h]
\begin{center}
\includegraphics[width=0.65\textwidth]{./3_results/2_2.eps}
\end{center}
\caption{График для $Tm(t)$ при $\tau=0.1$, $k=2$, $m=1$}
\end{figure}

\begin{figure}[h]
\begin{center}
\includegraphics[width=0.65\textwidth]{./3_results/2_3.eps}
\end{center}
\caption{Общий график при $\tau=0.1$, $k=2$, $m=1$}
\end{figure}

\newpage

\subsubsection{Приближение порядка $m=2$}

\begin{figure}[h]
\begin{center}
\includegraphics[width=0.65\textwidth]{./3_results/2_4.eps}
\end{center}
\caption{График для $Tm(t)$ при $\tau=0.1$, $k=2$, $m=2$}
\end{figure}

\begin{figure}[h]
\begin{center}
\includegraphics[width=0.65\textwidth]{./3_results/2_5.eps}
\end{center}
\caption{Общий график при $\tau=0.1$, $k=2$, $m=2$}
\end{figure}

\newpage

\subsubsection{Приближение порядка $m=3$}

\begin{figure}[h]
\begin{center}
\includegraphics[width=0.65\textwidth]{./3_results/2_6.eps}
\end{center}
\caption{График для $Tm(t)$ при $\tau=0.1$, $k=2$, $m=3$}
\end{figure}

\begin{figure}[h]
\begin{center}
\includegraphics[width=0.65\textwidth]{./3_results/2_7.eps}
\end{center}
\caption{Общий график при $\tau=0.1$, $k=2$, $m=3$}
\end{figure}

\newpage

\subsubsection{Приближение порядка $m=4$}

\begin{figure}[h]
\begin{center}
\includegraphics[width=0.65\textwidth]{./3_results/2_8.eps}
\end{center}
\caption{График для $Tm(t)$ при $\tau=0.1$, $k=2$, $m=4$}
\end{figure}

\begin{figure}[h]
\begin{center}
\includegraphics[width=0.65\textwidth]{./3_results/2_9.eps}
\end{center}
\caption{Общий график при $\tau=0.1$, $k=2$, $m=4$}
\end{figure}

\newpage

\subsubsection{Приближение порядка $m=5$}

\begin{figure}[h]
\begin{center}
\includegraphics[width=0.65\textwidth]{./3_results/2_10.eps}
\end{center}
\caption{График для $Tm(t)$ при $\tau=0.1$, $k=2$, $m=5$}
\end{figure}

\begin{figure}[h]
\begin{center}
\includegraphics[width=0.65\textwidth]{./3_results/2_11.eps}
\end{center}
\caption{Общий график при $\tau=0.1$, $k=2$, $m=5$}
\end{figure}

\newpage

\subsubsection{Приближение порядка $m=6$}

\begin{figure}[h]
\begin{center}
\includegraphics[width=0.65\textwidth]{./3_results/2_12.eps}
\end{center}
\caption{График для $Tm(t)$ при $\tau=0.1$, $k=2$, $m=6$}
\end{figure}

\begin{figure}[h]
\begin{center}
\includegraphics[width=0.65\textwidth]{./3_results/2_13.eps}
\end{center}
\caption{Общий график при $\tau=0.1$, $k=2$, $m=6$}
\end{figure}

\newpage

\subsection{Промежуточный вывод}

В [\ref{sec:graph_1}] и [\ref{sec:graph_2}] были рассмотрены корректно поставленные задачи Коши для уравнения (\ref{eq:result-ODE-delay}) и рассмотрены их первые $6$ приближений (\ref{eq:result-ODE-no-delay}).

На основании этих данных можно сделать следующие выводы:

\begin{enumerate}
    \item Увеличение порядка приближения $m$ не гарантирует лучшей аппроксимации решения (\ref{eq:result-ODE-delay}).
    \item Открыт вопрос о выборе начальных условий в задаче Коши для приближений: эвристический подход с выбором их как значений функции истории и её производных в начальной точке неоптимален.
    \item Открыт вопрос о временном интервале и его зависимости от запаздывания $\tau$, на которых приближения адекватны.
\end{enumerate}

Далее приведем модели с гораздо большими значениями $k$, иллюстрируюшие случаи некорректности задачи и пограничные к ним.

\newpage

\subsection{Графики решений при $\tau=0.001$, $k=39$}

\vfill

\begin{figure}[h]
\begin{center}
\includegraphics[width=0.9\textwidth]{./3_results/3_1.eps}
\end{center}
\caption{График для $T(t)$ при $\tau=0.001$, $k=39$, $m=1$}
\end{figure}

\vfill

\newpage

\subsubsection{Приближение порядка $m=1$}

\begin{figure}[h]
\begin{center}
\includegraphics[width=0.65\textwidth]{./3_results/3_2.eps}
\end{center}
\caption{График для $Tm(t)$ при $\tau=0.001$, $k=39$, $m=1$}
\end{figure}

\begin{figure}[h]
\begin{center}
\includegraphics[width=0.65\textwidth]{./3_results/3_3.eps}
\end{center}
\caption{Общий график при $\tau=0.001$, $k=39$, $m=1$}
\end{figure}

\newpage

\subsubsection{Приближение порядка $m=2$}

\begin{figure}[h]
\begin{center}
\includegraphics[width=0.65\textwidth]{./3_results/3_4.eps}
\end{center}
\caption{График для $Tm(t)$ при $\tau=0.001$, $k=39$, $m=2$}
\end{figure}

\begin{figure}[h]
\begin{center}
\includegraphics[width=0.65\textwidth]{./3_results/3_5.eps}
\end{center}
\caption{Общий график при $\tau=0.001$, $k=39$, $m=2$}
\end{figure}

\newpage

\subsubsection{Приближение порядка $m=3$}

\begin{figure}[h]
\begin{center}
\includegraphics[width=0.65\textwidth]{./3_results/3_6.eps}
\end{center}
\caption{График для $Tm(t)$ при $\tau=0.001$, $k=39$, $m=3$}
\end{figure}

\begin{figure}[h]
\begin{center}
\includegraphics[width=0.65\textwidth]{./3_results/3_7.eps}
\end{center}
\caption{Общий график при $\tau=0.001$, $k=39$, $m=3$}
\end{figure}

\newpage

\subsubsection{Приближение порядка $m=4$}

\begin{figure}[h]
\begin{center}
\includegraphics[width=0.65\textwidth]{./3_results/3_8.eps}
\end{center}
\caption{График для $Tm(t)$ при $\tau=0.001$, $k=39$, $m=4$}
\end{figure}

\begin{figure}[h]
\begin{center}
\includegraphics[width=0.65\textwidth]{./3_results/3_9.eps}
\end{center}
\caption{Общий график при $\tau=0.001$, $k=39$, $m=4$}
\end{figure}

\newpage

\subsubsection{Приближение порядка $m=5$}

\begin{figure}[h]
\begin{center}
\includegraphics[width=0.65\textwidth]{./3_results/3_10.eps}
\end{center}
\caption{График для $Tm(t)$ при $\tau=0.001$, $k=39$, $m=5$}
\end{figure}

\begin{figure}[h]
\begin{center}
\includegraphics[width=0.65\textwidth]{./3_results/3_11.eps}
\end{center}
\caption{Общий график при $\tau=0.001$, $k=39$, $m=5$}
\end{figure}

\newpage

\subsubsection{Приближение порядка $m=6$}

\begin{figure}[h]
\begin{center}
\includegraphics[width=0.65\textwidth]{./3_results/3_12.eps}
\end{center}
\caption{График для $Tm(t)$ при $\tau=0.001$, $k=39$, $m=6$}
\end{figure}

\begin{figure}[h]
\begin{center}
\includegraphics[width=0.65\textwidth]{./3_results/3_13.eps}
\end{center}
\caption{Общий график при $\tau=0.001$, $k=39$, $m=6$}
\end{figure}

\newpage

\subsection{Графики решений при $\tau=0.001$, $k=40$}

\vfill

\begin{figure}[h]
\begin{center}
\includegraphics[width=0.9\textwidth]{./3_results/4_1.eps}
\end{center}
\caption{График для $T(t)$ при $\tau=0.001$, $k=40$, $m=1$}
\end{figure}

\vfill

\newpage

\subsubsection{Приближение порядка $m=1$}

\begin{figure}[h]
\begin{center}
\includegraphics[width=0.65\textwidth]{./3_results/4_2.eps}
\end{center}
\caption{График для $Tm(t)$ при $\tau=0.001$, $k=40$, $m=1$}
\end{figure}

\begin{figure}[h]
\begin{center}
\includegraphics[width=0.65\textwidth]{./3_results/4_3.eps}
\end{center}
\caption{Общий график при $\tau=0.001$, $k=40$, $m=1$}
\end{figure}

\newpage

\subsubsection{Приближение порядка $m=2$}

\begin{figure}[h]
\begin{center}
\includegraphics[width=0.65\textwidth]{./3_results/4_4.eps}
\end{center}
\caption{График для $Tm(t)$ при $\tau=0.001$, $k=40$, $m=2$}
\end{figure}

\begin{figure}[h]
\begin{center}
\includegraphics[width=0.65\textwidth]{./3_results/4_5.eps}
\end{center}
\caption{Общий график при $\tau=0.001$, $k=40$, $m=2$}
\end{figure}

\newpage

\subsubsection{Приближение порядка $m=3$}

\begin{figure}[h]
\begin{center}
\includegraphics[width=0.65\textwidth]{./3_results/4_6.eps}
\end{center}
\caption{График для $Tm(t)$ при $\tau=0.001$, $k=40$, $m=3$}
\end{figure}

\begin{figure}[h]
\begin{center}
\includegraphics[width=0.65\textwidth]{./3_results/4_7.eps}
\end{center}
\caption{Общий график при $\tau=0.001$, $k=40$, $m=3$}
\end{figure}

\newpage

\subsubsection{Приближение порядка $m=4$}

\begin{figure}[h]
\begin{center}
\includegraphics[width=0.65\textwidth]{./3_results/4_8.eps}
\end{center}
\caption{График для $Tm(t)$ при $\tau=0.001$, $k=40$, $m=4$}
\end{figure}

\begin{figure}[h]
\begin{center}
\includegraphics[width=0.65\textwidth]{./3_results/4_9.eps}
\end{center}
\caption{Общий график при $\tau=0.001$, $k=40$, $m=4$}
\end{figure}

\newpage

\subsubsection{Приближение порядка $m=5$}

\begin{figure}[h]
\begin{center}
\includegraphics[width=0.65\textwidth]{./3_results/4_10.eps}
\end{center}
\caption{График для $Tm(t)$ при $\tau=0.001$, $k=40$, $m=5$}
\end{figure}

\begin{figure}[h]
\begin{center}
\includegraphics[width=0.65\textwidth]{./3_results/4_11.eps}
\end{center}
\caption{Общий график при $\tau=0.001$, $k=40$, $m=5$}
\end{figure}

\newpage

\subsubsection{Приближение порядка $m=6$}

\begin{figure}[h]
\begin{center}
\includegraphics[width=0.65\textwidth]{./3_results/4_12.eps}
\end{center}
\caption{График для $Tm(t)$ при $\tau=0.001$, $k=40$, $m=6$}
\end{figure}

\begin{figure}[h]
\begin{center}
\includegraphics[width=0.65\textwidth]{./3_results/4_13.eps}
\end{center}
\caption{Общий график при $\tau=0.001$, $k=40$, $m=6$}
\end{figure}

\newpage

\subsection{Графики решений при $\tau=0.001$, $\gamma = \frac{\pi}{2*0.001}$}

В ходе исследования было сделано предположение о том, что необходимым условием устойчивости исходной задачи с запаздыванием является $\gamma \tau < \dfrac{\pi}{2}$. Построим решение и его приближения, например, в критическом случае $\tau=0.001$, $\gamma = \frac{\pi}{2*0.001}$.

\vfill

\begin{figure}[h]
\begin{center}
\includegraphics[width=0.9\textwidth]{./3_results/5_1.eps}
\end{center}
\caption{График для $T(t)$ при $\tau=0.001$, $\gamma = \frac{\pi}{2*0.001}$, $m=1$}
\end{figure}

\vfill

\newpage

\subsubsection{Приближение порядка $m=1$}

\begin{figure}[h]
\begin{center}
\includegraphics[width=0.65\textwidth]{./3_results/5_2.eps}
\end{center}
\caption{График для $Tm(t)$ при $\tau=0.001$, $\gamma = \frac{\pi}{2*0.001}$, $m=1$}
\end{figure}

\begin{figure}[h]
\begin{center}
\includegraphics[width=0.65\textwidth]{./3_results/5_3.eps}
\end{center}
\caption{Общий график при $\tau=0.001$, $\gamma = \frac{\pi}{2*0.001}$, $m=1$}
\end{figure}

\newpage

\subsubsection{Приближение порядка $m=2$}

\begin{figure}[h]
\begin{center}
\includegraphics[width=0.65\textwidth]{./3_results/5_4.eps}
\end{center}
\caption{График для $Tm(t)$ при $\tau=0.001$, $\gamma = \frac{\pi}{2*0.001}$, $m=2$}
\end{figure}

\begin{figure}[h]
\begin{center}
\includegraphics[width=0.65\textwidth]{./3_results/5_5.eps}
\end{center}
\caption{Общий график при $\tau=0.001$, $\gamma = \frac{\pi}{2*0.001}$, $m=2$}
\end{figure}

\newpage

\subsubsection{Приближение порядка $m=3$}

\begin{figure}[h]
\begin{center}
\includegraphics[width=0.65\textwidth]{./3_results/5_6.eps}
\end{center}
\caption{График для $Tm(t)$ при $\tau=0.001$, $\gamma = \frac{\pi}{2*0.001}$, $m=3$}
\end{figure}

\begin{figure}[h]
\begin{center}
\includegraphics[width=0.65\textwidth]{./3_results/5_7.eps}
\end{center}
\caption{Общий график при $\tau=0.001$, $\gamma = \frac{\pi}{2*0.001}$, $m=3$}
\end{figure}

\newpage

\subsubsection{Приближение порядка $m=4$}

\begin{figure}[h]
\begin{center}
\includegraphics[width=0.65\textwidth]{./3_results/5_8.eps}
\end{center}
\caption{График для $Tm(t)$ при $\tau=0.001$, $\gamma = \frac{\pi}{2*0.001}$, $m=4$}
\end{figure}

\begin{figure}[h]
\begin{center}
\includegraphics[width=0.65\textwidth]{./3_results/5_9.eps}
\end{center}
\caption{Общий график при $\tau=0.001$, $\gamma = \frac{\pi}{2*0.001}$, $m=4$}
\end{figure}

\newpage

\subsubsection{Приближение порядка $m=5$}

\begin{figure}[h]
\begin{center}
\includegraphics[width=0.65\textwidth]{./3_results/5_10.eps}
\end{center}
\caption{График для $Tm(t)$ при $\tau=0.001$, $\gamma = \frac{\pi}{2*0.001}$, $m=5$}
\end{figure}

\begin{figure}[h]
\begin{center}
\includegraphics[width=0.65\textwidth]{./3_results/5_11.eps}
\end{center}
\caption{Общий график при $\tau=0.001$, $\gamma = \frac{\pi}{2*0.001}$, $m=5$}
\end{figure}

\newpage

\subsubsection{Приближение порядка $m=6$}

\begin{figure}[h]
\begin{center}
\includegraphics[width=0.65\textwidth]{./3_results/5_12.eps}
\end{center}
\caption{График для $Tm(t)$ при $\tau=0.001$, $\gamma = \frac{\pi}{2*0.001}$, $m=6$}
\end{figure}

\begin{figure}[h]
\begin{center}
\includegraphics[width=0.65\textwidth]{./3_results/5_13.eps}
\end{center}
\caption{Общий график при $\tau=0.001$, $\gamma = \frac{\pi}{2*0.001}$, $m=6$}
\end{figure}