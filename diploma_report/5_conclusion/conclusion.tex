\section*{Заключение}
\addcontentsline{toc}{section}{Заключение}

В настоящей работе было рассмотрено дифференциальное уравнение теплопроводности с запаздыванием

\begin{equation*}
\rho_t(\bar{x},t) = -a^2 \Delta \rho(\bar{x},t-\tau) + \rho(\bar{x},t-\tau) div \bar{V}(\bar{x},t-\tau) + (\bar{V}(\bar{x},t-\tau),\nabla \rho(\bar{x},t-\tau)),
\end{equation*}

полученное из системы уравнения Лиувилля и закона Фурье с запаздыванием по времени:

\begin{equation*}
\left\{
\begin{aligned}
\rho_t(\bar{x},t) + div \bar{J}(\bar{x},t) = f(\bar{x},t),\\
\bar{J}(\bar{x},t+\tau) = -a^2 \nabla \rho(\bar{x},t) + \bar{V}(\bar{x},t) \rho(\bar{x},t).
\end{aligned}
\right.
\end{equation*}

В предположениях

\begin{enumerate}
\item одномерной задачи,
\item $\rho (x,t) = e^{\frac{V}{2a^2} x} u(x,t)$,
\item $\sum\limits_{n=0}^{m} \dfrac{\partial^n f}{\partial t^n} \dfrac{\tau^n}{n!} = g(x,t) \equiv 0 \Leftarrow f(x,t) \equiv 0$,
\item $V=const$
\end{enumerate}

задача была сведена к исследованию уравнения реакции-диффузии с запаздыванием

\begin{equation*}
u_t (x,t) - a^2 u_{xx} (x,t-\tau) + \dfrac{V^2}{4a^2} u (x,t-\tau) = 0.
\end{equation*}

Для него были построены приближения

\begin{equation*}
\sum\limits_{n=0}^{m} \dfrac{\partial^{n+1} u}{\partial t^{n+1}} \dfrac{\tau^n}{n!} - a^2 \dfrac{\partial^2 u}{\partial x^2} + \dfrac{V^2}{4a^2} u = 0
\end{equation*}

и проверена устойчивость их решений:

\begin{enumerate}
\item при $m=1$ \--- устойчиво при любых значениях параметров,
\item при $2 < m < 6$ \--- могут быть как устойчивы, так и нет в зависимости от $\gamma$ и $\tau$,
\item при $m=6$ \--- всегда неустойчивы.
\end{enumerate}

Также на устойчивость было исследовано само уравнение реакции-диффузии путем разложения методом Фурье и получено неустойчивость его решений при любых значениях параметров.

Однако, решения вспомогательных одномерных уравнений с запаздыванием

\begin{equation*}
\tau T'_k(t) + \gamma(k^2) T_k(t-\tau) = 0
\end{equation*}

могли иметь различный характер в завимости от параметров: был доказано утверждение о том, что $\gamma \tau < 1$ \--- достаточное условие устойчивости, и на основании численного моделирования было сделано предположение, что $\gamma \tau < \frac{\pi}{2}$ \--- необходимое условие устойчивости.

Также был исследован вопрос о корретности постановок задач Коши для приближений и установлено, что естественный метод приравнивания начального условия к значению функции истории и её производных в начальной точке 

\begin{equation*}
T^{(k)}(t_0) = I^{(k)}(t)|_{t_0}
\end{equation*}

не всегда приводит к лучшему приближению.

Также было установлено, что рост $m$, вообще говоря, не приводит к улучшению приближения исходной задачи.

Просуммировав все вышеизложенное можно сделать вывод о том, что ествественное приближение уравнений с отклонением аргумента (в частности, с запаздыванием) моделями обыкновенных дифференциальных уравнений представляют из себя нетривиальную задачу и должно быть исследовано индивидуально для каждой подобной поставленной задачи в силу непредсказаумости асимптотического поведения решений уравнений с запаздыванием.

Также стоит отметить, что подобные задачи все чаще появляются в приложениях различных дисциплин (как, например, модель Калецкого в экономике) и последующее развитие теории уравнений дифференциальных уравнений с отклоняющимся аргументом необходимо для дальнейшего продвижения подобных моделей во всех областях науки.