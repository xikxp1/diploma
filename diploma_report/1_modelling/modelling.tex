\section{Изучение поставленной задачи с помощью\\компьютерного моделирования}

\subsection{Математическая постановка исходной задачи}

Запишем \textbf{общее уравнение непрерывности} в дифференциальной форме:

\begin{equation}\label{eq:main}
\rho_t(\bar{x},t) + div \bar{J}(\bar{x},t) = f(\bar{x},t)
\end{equation}

Оно представляет собой сильную (локальную) форму закона сохранения и выражает связь между потоком $\bar{J} (\bar{x},t)$ и концетрацией (температурой) $\rho(\bar{x},t)$.

Чтобы корректно поставить задачу решения (\ref{eq:main}) необходимо дополнить его ещё одним условием \--- в данной работе оно представлено \textbf{законом диффузии с конвекцией} с отклоняющимся аргументом:

\begin{equation}\label{eq:diffusion-convection}
\bar{J}(\bar{x},t+\tau) = -a^2 \nabla \rho(\bar{x},t) + \bar{V}(\bar{x},t) \rho(\bar{x},t)
\end{equation}

В дальнейшем уравнение (\ref{eq:main}) будет классифицировано с точки зрения теории дифференциальных уравнений с отклоняющимся аргументом. Предварительно приведем необходимую для исследования информацию.

\subsection{Теория дифференциальных уравнений\\с отклоняющимся аргументом}

\subsubsection{Базовые понятия и определения}

\textbf{Дифференциальными уравнениями с отклоняющимся аргументом} называются дифференциальные уравнения, в которые неизвестная функция и её проиводные входят, вообще говоря, при различных значениях аргумента.

Хотя уравнения с отклоняющимся аргументом по виду очень похожи на обыкновенные дифференциальные уравнения, факт отклонения аргумента усложняет их анализ.

Рассмотрим простейший пример

\begin{equation}\label{eq:example}
\dot{x}(t) = f(t,x(t),x(t-\tau)),
\end{equation}

где $\tau$ \--- положительная константа.

Для начала отметим, что для решения задачи Коши уже недостаточно задать одно начальное условие $x(t_0)=x_0$, ведь необходимо знать исходные значения на всем отрезке $[t_0-\tau,t_0]$, т.е задать функцию $x_0(t)$, определенную на отрезке $[t_0-\tau,t_0]$ \--- \textbf{начальную функцию} (в зарубежной литературе \textbf{history function} или \textbf{initial function}).

\subsubsection{Простейшая модель с запаздыванием}

Для примера рассмотрим две задачи Коши: одну для обыкновенного дифференциального уравнения, другую \--- для уравнения с отклонением по аргументу:

\begin{equation}\label{eq:no-delay}
\left\{
\begin{aligned}
x'(t) = x(t), \quad t>0\\
x(0) = 1
\end{aligned}
\right.
\end{equation}

\begin{equation}\label{eq:delay}
\left\{
\begin{aligned}
y'(t) = y(t-1), \quad t>0\\
y(t) = 1, \quad -1 \leq t \leq 0
\end{aligned}
\right.
\end{equation}

Методы решения подобных уравнений с запаздыванием будут рассмотрены в работе позже, а пока построим графики этих решений:

\begin{center}
\includegraphics[width=0.65\textwidth]{./1_modelling/comparison.eps}
\end{center}

Наглядно видно, что решения различны. Позже будет приведено аналитическое решение задачи (\ref{eq:delay}) и доказано, что $y(t)$ всегда возрастает медленнее экспоненциального решения, т.е. $x(t)$.

Также решение зависит от начальной функции: для примера рассмотрим такую задачу:

\begin{equation}\label{eq:no-delay-history}
\left\{
\begin{aligned}
y'_h(t) = y_h(t-1), \quad t>0\\
y_h(t) = 1+t, \quad -1 \leq t \leq 0
\end{aligned}
\right.
\end{equation}

Отметим, что $y_h(0) = y(0) = 1$.

Построим решения на одном графике:

\begin{center}
\includegraphics[width=0.65\textwidth]{./1_modelling/comparison_history.eps}
\end{center}

Таким образом, начальная функция определяет характер решения задачи.

\subsubsection{Классификация моделей с отклоняющимся аргументом}

Рассотрим обобщенное уранвнения $n$-ого порядка c $l$ отклонениями аргумента (вообще говоря, отклонение может быть непостоянным и зависеть, например, от значений самого аргумента).

\begin{equation}\label{eq:general-delay}
\begin{aligned}
x^{(m_0)}(t) = f(t,x(t),\dots,x^{(m_0-1)}(t),x(t-\tau_1(t)),\dots,\\\dots,x^{(m_1)}(t-\tau_1(t)),\dots,x(t-\tau_l(t)),\dots,x^{(m_l)}(t-\tau_l(t)))
\end{aligned}
\end{equation}

Здесь $\tau_i(t) \geq 0$.

Каменским Г.А. была введена естественная классификация (\ref{eq:general-delay}). Обозначим $\lambda = m_0 - \max\limits_{1 \leq i \leq l} m_i$.

\begin{enumerate}
\item Если $\lambda > 0$, то такое уравнения называется уравнением с \textbf{запаздывающим аргументом};
\item Если $\lambda = 0$, \--- уравнением \textbf{нейтрального типа};
\item Если $\lambda < 0$, \--- уравнением \textbf{опережающего типа}.
\end{enumerate}

Вернемся к исходной задаче (\ref{eq:main}, \ref{eq:diffusion-convection}):

\begin{equation}
\left\{
\begin{aligned}
\rho_t(\bar{x},t) + div \bar{J}(\bar{x},t) = f(\bar{x},t),\\
\bar{J}(\bar{x},t+\tau) = -a^2 \nabla \rho(\bar{x},t) + \bar{V}(\bar{x},t) \rho(\bar{x},t)
\end{aligned}
\right.
\end{equation}

Объединим в одно уравнение относительно $\rho(x,t)$:

\begin{equation}\label{eq:main-one}
\rho_t(\bar{x},t) - -a^2 \Delta \rho(\bar{x},t-\tau) + \rho(\bar{x},t-\tau) div \bar{V}(\bar{x},t-\tau) + (\bar{V}(\bar{x},t-\tau),\nabla \rho(\bar{x},t-\tau))
\end{equation}

Отметим, что дивергенция как дифференциальный оператор была применена по пространственным переменным $bar{x}$. Из этого следует, что в левой части (\ref{eq:main-one}) не содержится производных по $t$ и уравнение классифицируется как уравнение с запаздывающим аргументом при $\tau>0$ и как опережающего типа при $\tau<0$.

\subsubsection{Метод шагов решения уравнений\\с отклоняющимся аргументом}

Метод шагов \--- естественный метод решения уравнений с отклоняющимся аргуентом. Рассмотрим простейший пример:

\begin{equation}
x'(t) = f(t,x(t),x(t-\tau)), \quad \tau>0
\end{equation}

Поставим задачу Коши для него:

\begin{equation}\label{eq:delayCauchy}
\left\{
\begin{aligned}
x'(t) = f(t,x(t),x(t-\tau)), \quad t>0,\\
x(t) = x_0(t), \quad -\tau \leq t \leq 0
\end{aligned}
\right.
\end{equation}

Тогда на отрезке $[0,\tau]$:

\begin{equation}\label{eq:toCauchy}
\left\{
\begin{aligned}
x'(t) = f(t,x(t),x_0(t-\tau)), \quad 0 \leq t \leq \tau,\\
x(0) = x_0(0)
\end{aligned}
\right.
\end{equation}

Отметим, что \ref{eq:toCauchy} \--- задача Коши уже для обыкновенного дифференциального уравнения. Предположим, что $x_1(t)$ \--- ее решение. Тогда на отрезке $[\tau,2\tau]$:

\begin{equation}\label{eq:toCauchy}
\left\{
\begin{aligned}
x'(t) = f(t,x(t),x_1(t-\tau)), \quad \tau \leq t \leq 2\tau,\\
x(\tau) = x_1(\tau)
\end{aligned}
\right.
\end{equation}

Продолжая подобные рассуждения, можно найти решение \ref{eq:delayCauchy}. Отметим, что решать можно как аналитически, так и численно.

К минусам этого метода можно отнести зависимость от величины отклонения и сложность адаптации к общему случаю зависимости запаздывания от аргумента.

\subsubsection{Некоторые сведения о решениях линейных уравнений\\с постоянными коэффициентами и постоянными\\отклонениями аргумента}

Рассмотрим линейное однородное уравнение с постоянными коэффициентами и постоянными отклонениями аргумента

\begin{equation}
\sum\limits_{i=0}^{n} \sum\limits_{j=0}^{l} a_{ij} x^{(i)}(t-\tau_j) = 0
\end{equation}