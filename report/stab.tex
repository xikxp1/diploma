\section{Исследование устойчивости исходного уравнения в частных производных}

\subsection{Метод Фурье для исходного уравнения}

Для уравнения \ref{eq:final} применим \textbf{метод Фурье разделения переменных}:

\begin{equation*}
u(x,t) = T(t) X(x)
\end{equation*}

\begin{equation*}
\sum\limits_{k=0}^{m} \dfrac{\tau^k}{k!} T^{(k+1)} X + \dfrac{V^2}{4a^2} T X =a^2 T X''
\end{equation*}

\begin{equation*}
\dfrac{\sum\limits_{k=0}^{m} \dfrac{\tau^k}{k!} T^{(k+1)} + \dfrac{V^2}{4a^2} T}{a^2 T} = \dfrac{X''}{X} = -\lambda = -k^2
\end{equation*}

\begin{equation}
\left\{
\begin{aligned}
\sum\limits_{k=0}^{m} \dfrac{\tau^k}{k!} T^{(k+1)} + \underbrace{ \left( \dfrac{V^2}{4a^2} + a^2 k^2 \right)}_{\gamma} T & = 0\\
X'' + k^2 X & = 0
\end{aligned}
\right.
\end{equation}

Как видно, малый параметр входит только в уранение относительно $T(t)$.

Целью нашего дальнейшего исследования является обосновать устойчивость или неустойчивость решений 

\begin{equation}
\sum\limits_{k=0}^{m} \dfrac{\tau^k}{k!} T^{(k+1)} + \underbrace{ \left( \dfrac{V^2}{4a^2} + a^2 k^2 \right)}_{\gamma} T = 0
\end{equation}

при различных $m$.

\subsection{Исследование устойчивости обыкновенного дифференциального уравнения относительно $T(t)$}