\begin{center}
\textsc{\textbf{Особенности решений уравнений теплопроводности с отклоняющимся аргументом\\}}
\end{center}

Запишем \textbf{общее уравнение непрерывности} в дифференциальной форме:

\begin{equation}\label{eq:main}
\rho_t + div \bar{J} = f
\end{equation}

Оно представляет собой сильную (локальную) форму закона сохранения.

Рассмотрим случай, когда плотность потока будет выражена через плотность заряда следующим образом:

\begin{equation}
\bar{J} = -a^2 \nabla \rho + \bar{V} \rho
\end{equation}

Рассмотрим возмущенную систему, где $\tau$ \--- малый параметр:

\begin{equation*}
\bar{J}(x,t+\tau) = -a^2 \nabla \rho(x,t) + \bar{V}(x,t) \rho(x,t)
\end{equation*}

Разложим левую часть данного уравнения в ряд Тэйлора до члена порядка $m$:

\begin{equation*}
\bar{J}(x,t) + \bar{J_t}(x,t) \tau + \dots \dfrac{\partial^m \bar{J}}{\partial t^m}(x,t) \dfrac{\tau^m}{m!} = -a^2 \nabla \rho(x,t) + \bar{V}(x,t) \rho(x,t)
\end{equation*}

\begin{equation}\label{eq:sub}
\sum\limits_{k=0}^{m} \dfrac{\partial^k \bar{J}}{\partial t^k}(x,t) \dfrac{\tau^k}{k!} = -a^2 \nabla \rho(x,t) + \bar{V}(x,t) \rho(x,t)
\end{equation}

Продифференцировав исходное уравнение \ref{eq:main} $k$ раз, получим

\begin{equation*}
\dfrac{\partial^{k+1} \rho}{\partial t^{k+1}} + div \dfrac{\partial^k \bar{J}}{\partial t^k} = \dfrac{\partial^k f}{\partial t^k}
\end{equation*}

Домножим последнее уравнение на $\frac{\tau^k}{k!}$, перепишем его в следующем виде:

\begin{equation*}
\dfrac{\partial^{k+1} \rho}{\partial t^{k+1}} \dfrac{\tau^k}{k!} = div \dfrac{\partial^k \bar{J}}{\partial t^k} = \dfrac{\partial^k f}{\partial t^k}
\end{equation*}

%TODO: equation (7)
