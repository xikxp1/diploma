\documentclass[a4paper, 12pt]{article}

% PDF search & cut'n'paste
\usepackage{cmap}

\usepackage[T2A]{fontenc}
\usepackage[utf8] {inputenc}
\usepackage[english, russian] {babel}
\usepackage{indentfirst}

\usepackage{pscyr}
\renewcommand{\rmdefault}{ftm}

\usepackage{amsmath}
\usepackage{amsfonts}
\usepackage{amssymb}

\usepackage{enumerate}

\usepackage{graphicx}
\usepackage{epstopdf}

\usepackage{geometry}
\geometry{left=3cm}
\geometry{right=1.5cm}
\geometry{top=2cm}
\geometry{bottom=2cm}

\usepackage{setspace}

% Allow landscape pages for graphics, call like:
%
%       \afterpage{\clearpage
%       \begin{landscape}
%       \begin{figure}[p]
%       ...
%       \end{figure}
%       \end{landscape}
%       }
\usepackage{pdflscape}

% Provides support for setting the spacing between lines in a document. Package
% options include singlespacing, onehalfspacing, and doublespacing.
% http://www.ctan.org/tex-archive/macros/ ... /setspace/
\usepackage{setspace}

% This declaration makes TeX less fussy about line breaking. This can
% prevent overfull boxes, but may leave too much space between words.
% As this really isn't a fine art typography, we'll turn it on, so
% we won't have paragraphs which spans on the margins...
\sloppy