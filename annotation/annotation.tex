\documentclass[a4paper, 12pt]{article}
\usepackage[T2A]{fontenc}
\usepackage[utf8] {inputenc}
\usepackage[english, russian] {babel}
\usepackage{indentfirst}
\usepackage{amsfonts}
\usepackage{amsmath}

\usepackage{geometry}
\geometry{left=2cm}
\geometry{right=1.5cm}
\geometry{top=1cm}
\geometry{bottom=2cm}

\usepackage{fancyhdr}
\pagestyle{fancy}
\fancyhead{}
\fancyhead[R]{}
\fancyhead[L]{\textit{Особенности математических моделей теплопроводности с отклоняющимся аргументом}}
\setlength{\headheight}{36pt}
\fancyfoot{}

\begin{document}

\textbf{Тема дипломного проекта}: Особенности математических моделей теплопроводности с отклоняющимся аргументом\\

\textbf{Автор}: Качалкин Иван, УПМ-411\\

\textbf{Руководитель}: проф., д.ф.-м.н. Филимонов А.М.\\

Данный дипломный проект посвящен изучению математических моделей теплопроводности с отклоняющимся аргументом.

В нем исследовано уравнение локального закона сохранения (Лиувилля) с запаздыванием $\tau$:

\begin{equation}
\left\{
\begin{aligned}
\rho_t(x,t) + div \bar{J}(x,t) = f(x,t),\\
\bar{J}(x,t+\tau) = -a^2 \nabla \rho(x,t) + \bar{V}(x,t) \rho(x,t)
\end{aligned}
\right.
\end{equation}

Для него был рассмотрен вопрос устойчивости, а также построены приближения моделями без запаздывания и произведена оценка их адекватности, а также доказаны теоремы об устойчивости.

\end{document}